\begin{titleDef}{Metrische Räume}
\label{MetrischerRaum}
Ein Paar (X,d) aus einer Menge X und einer \hyperref[Metrik]{Metrik} heißt metrischer Raum.\\
Metrische Räume sind \hyperref[hausdorffsch]{hausdorffsch}.
\end{titleDef}

\begin{titleDef}{Topologische Räume}
\label{Topologie}
Ein topologischer Raum ist ein Paar $(X,\mathcal{O})$ bestehend aus einer Menge X und einem System $\mathcal{O}\subseteq\mathcal{P}(X)$ von Teilmengen von X, so dass gilt
\begin{enumerate}
    \item $X,\emptyset\in\mathcal{O}$
    \item Der Durchschnitt von endlich vielen Mengen aus $\mathcal{O}$ ist wieder in $\mathcal{O}$$$\Leftrightarrow\bigcap^n_{i=1}X_i\in\mathcal{O}, n\in\mathbb{N},X_i\in\mathcal{O}$$
    \item Die Vereinigung von beliebig vielen Mengen aus $\mathcal{O}$ ist wieder in $\mathcal{O}$\\$$\Leftrightarrow\bigcup_{i=1}^kX_i\in\mathcal{O}, k\in\mathbb{N}\cup\{\infty\},X_i\in\mathcal{O}$$
\end{enumerate}
Ein solches System $\mathcal{O}$ von Teilmengen heißt \textbf{Topologie} von X 
\end{titleDef}

\begin{titleDef}{Umgebung}
\label{Umgebung}
Sei $B\subset X$ eine Teilmenge von X einer \Toporef \Topo. Eine Teilmenge ${U\subseteq X}$ heißt Umgebung von B, falls eine offene Menge O bezüglich der \Toporef existiert, so dass $B\subseteq O\subseteq U$. Also B ist komplett in einer offenen Teilmenge von X enthalten. Beachte das B nicht offen sein muss.
\end{titleDef}

\begin{titleDef}{Topologische Mannigfaltigkeiten}
\label{Mannigfaltigkeit}
Eine \textbf{(topologische) Mannigfaltigkeit} ist ein \Toporeflong $M$ mit:
\listbsp
\begin{enumerate}
	\item $M$ ist \textbf{lokal euklidisch}, d.h für alle Punkte $p\in M$ existiert eine \hyperref[offen]{offene} \hyperref[Umgebung]{Umgebung} $U$ von $p$ und ein \hyperref[homoemorph]{Homöomorphismus} ${\varphi:U\to \varphi(U)\subseteq\mathbb{R}^n}$ auf eine offene Teilmenge von $\mathbb{R}^n$ für gewisses n.
	\item $M$ ist \hyperref[hausdorffsch]{Hausdorffsch} und hat eine abzählbare \hyperref[basisTopo]{Basis}.
\end{enumerate}
\end{titleDef}

\begin{titleDef}{Differenzierbare Mannigfaltigkeit}
\label{diffMannigfaltigkeit}
Eine \textbf{Differenzierbare Mannigfaltigkeit} ist eine topologische Mannigfaltigkeit versehen mit einem \hyperref[maxAtlas]{maximalen Atlas}.
\end{titleDef}

\begin{titleDef}{Simplizialkomplexe}
\label{simplex}
Ein \textbf{k-Simplex} in $\mathbb{R}^n$ ist die \hyperref[konvHuelle]{konvexe Hülle} $s(v_0,v_1,\ldots,v_k)$ von k+1 affin unabhängigen Punkten $v_0,v_1,\ldots,v_k$.\par
Eine endliche Menge $K$ von Simplexen heißt (endlicher) $\textbf{Simplizial-Komplex}$ wenn gilt:
\begin{enumerate}
	\item Mit jedem Simplex in K enthält K auch alle zugehörigen Teilsimplexe.\\Also wenn das "Dreieck" $s(v_0,v_1,v_2)\in K$ dann gilt: Die Seiten $s(v_0,v_1)\in K,s(v_0,v_2)\in K,s(v_1,v_2)\in K$ sowie die Punkte $s(v_0)\in K,s(v_1)\in K,s(v_2)\in K$ liegen in K.
	\item Der Durchschnitt von zwei Simplexen die in K liegen ist entweder leer oder \textbf{genau} ein gemeinsames Teilsimplex.\\
	Also schneiden sich zwei Simplexe entweder gar nicht, in genau einem Punkt oder genau einer vollständigen Seite, usw
\end{enumerate}
Simplizialkomplexe entstehen durch \hyperref[verklebung]{verkleben} von Simplexen.
\end{titleDef}

\begin{titleDef}{konvexe Polyeder}
\label{konvexPoly}
Eine Teilmenge $P\subset\mathbb{R}^3$ heißt \textbf{konvexes Polyeder}, falls
\begin{enumerate}
	\item $P$ ist Durchschnitt von endlich vielen affinen Halbräumen\\
	Ein affiner Halbraum kann man sich als die linke oder rechte Hälfte einer Ebene durch den $\mathbb{R}^3$ vorstellen.
	\item $P$ ist beschränkt und nicht ganz in einer Ebene enthalten.
	\item $P$ ist konvex, d.h. mit je zwei Punkten $p,q\in P$ liegt auch das Geradensegment $\overline{pq}$ ganz in $P$.
\end{enumerate}
Eigenschaft 3. ist nicht notwendig für einen konvexen Polyeder gilt aber wenn eine Teilmenge ein konvexes Polyeder ist.\par
Ein Polyeder heißt (m,n)-\textbf{regulär}, falls alle Seitenflächen des Polyeders n-Gone sind also alle Kanten gleich lang sind und in jeder Ecke genau m solcher n-Gone /Kanten zusammentreffen.
\end{titleDef}

\begin{titleDef}{Reguläre Flächen}
\label{regFlaeche}
Eine Teilmenge $S\subset\mathbb{R}^3$ heißt \textbf{reguläre Fläche}, falls es zu jedem Punkt $p\in S$ eine \hyperref[offen]{offene} \hyperref[Umgebung]{Umgebung} $V$ um $p$ in $\mathbb{R}^3$, eine \hyperref[offen]{offene} \hyperref[Umgebung]{Umgebung} $U\subset\mathbb{R}^2$ und eine $C^\infty$-Abbildung die sogenannte \textbf{(lokale) Parametrisierung}
$$x:U\to\mathbb{R}^3;\: (u,v)\mapsto x(u,v)=(x_1(u,v),x_2(u,v),x_3(u,v))$$
gibt, so dass
\begin{enumerate}[label=(\arabic*)]
	\item $x(U)=S\cap V$ und $x:U\to S\cap V$ ist ein \hyperref[homoemorph]{Homöomorphismus} 
	\item Die \hyperref[differenzial]{Ableitung} $dx(u,v):T_{(u,v)}\mathbb{R}^2\to T_{x(u,v)}\mathbb{R}^3$ ist injektiv für $(u,v)\in U$
\end{enumerate}
Bedingung (2) ist äquivalent zu
\begin{enumerate}[label=(2\alph*)]
	\item Die \hyperref[funktmatrix]{Funktionalmatrix} $\begin{pmatrix}
		\frac{\partial x_1}{\partial u}(u,v)&\frac{\partial x_1}{\partial v}(u,v)\\
		\frac{\partial x_2}{\partial u}(u,v)&\frac{\partial x_2}{\partial v}(u,v)\\
		\frac{\partial x_3}{\partial u}(u,v)&\frac{\partial x_3}{\partial v}(u,v)\\
		\end{pmatrix}$ hat Rang 2 für $(u,v)\in U$
	\item Die Spaltenvektoren $x_u(u,v),\ x_v(u,v)$ sind linear unabhängig.
	\item Das \hyperref[vektorprodukt]{Vektorprodukt} $x_u(u,v)\wedge x_v(u,v)\neq0$ für alle $(u,v)\in U$
\end{enumerate}
\end{titleDef}

\begin{titleDef}{Polygon}
\label{polygon}
\label{innenwinkel}
Sei $S$ eine \hyperref[regFlaeche]{reguläre Fläche}. Ein \textbf{Polygon} in $S$ ist eine Teilmenge $G$, die \hyperref[homoemorph]{homöomorph} zu einer \hyperref[abgeschlossen]{abgeschlossenen} Kreisscheibe ist und deren Rand $\partial G$ das Bild einer \hyperref[einfachgeschlossen]{einfach geschlossenen}, stückweise \hyperref[regulaer]{regulären} Kurve ist. Die \textbf{Innenwinkel} des Polygons sind durch $\alpha_i=\pi-\delta_i$ mit den \hyperref[aussenwinkel]{Außenwinkeln} $\delta_i$
\end{titleDef}

\begin{titleDef}{geodätische Dreiecke}
\label{geodaetischeDreiecke}
Ein \textbf{geodätisches Dreieck} in einer \hyperref[regFlaeche]{regulären Fläche} S ist ein \hyperref[polygon]{Polygon} $\Delta$, das genau drei Ecken hat und dessen drei Randsegmente/Kanten \hyperref[geodaetische]{Geodätische} sind.\par
Seien $\alpha,\beta,\gamma$ die Innenwinkel des geodätischen Dreiecks. Für die Innenwinkelsumme eines geodätischen Dreiecks in einer Fläche $S$ die konstante \hyperref[gausskruemmung]{Gaußkrümmung K} haben gilt:
\begin{itemize}
	\item Für $K\equiv0\Longrightarrow \alpha+\beta+\gamma=\pi$
	\item Für $K\equiv1\Longrightarrow \alpha+\beta+\gamma>\pi$
	\item Für $K\equiv-1\Longrightarrow \alpha+\beta+\gamma<\pi$
\end{itemize}
\end{titleDef}

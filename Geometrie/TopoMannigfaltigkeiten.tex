\begin{rawDef}
\begin{enumerate}
	\item Abzählbar viele Punkte versehen mit der \hyperref[diskTopo]{diskreten Topologie} bilden ein 0-dimensionale Mannigfaltigkeit
	\item Der \hyperref[Einheitskreis]{Einheitskreis} $S^1$ ist eine \hyperref[kompakt]{kompakte}~\hyperref[zusammenhang]{ zusammenhängende} 1-dimensionale Mannigfaltigkeit.
	\item Die reellen Zahlen $\mathbb{R}$ bilden eine \hyperref[kompakt]{nicht kompakte}~\hyperref[zusammenhang]{zusammenhängende} 1-dimensionale Mannigfaltigkeit.
	\item Jede \hyperref[offen]{offene} Teilmenge von $\mathbb{R}^n$ (insbesondere $\mathbb{R}^n$ selbst) ist eine n-dimensionale Mannigfaltigkeit.
	\item Jede \hyperref[offen]{offene} Teilmenge einer n-dimensionalen Mannigfaltigkeit ist selbst wieder eine n-dimensionale Mannigfaltigkeit.
	\item Die allgemeine lineare Gruppe \\$GL(n,\mathbb{R})=\{A\in\mathbb{R}^{n\times n} |\ det(A)\neq0\}=det^{-1}(\mathbb{R}\textbackslash\{0\})$\\
	ist offen in $\mathbb{R}^{n\times n}$ also eine $n\times n$-dimensionale Mannigfaltigkeit.
	\item Die \hyperref[ndimsphere]{n-dimensionale Einheitssphäre} $S^n_1=S^n$ ist eine n-dimensionale Mannigfaltigkeit (ebenso beliebiege n-dimensionale Sphären $S^n_R$).
	\item Das Produkt einer m-dimensionalen Mannigfaltigkeit $M$mit einer n-dimensionalen Mannigfaltigkeit $N$ ist eine m+n-dimensionale Mannigfaltigkeit $M\times N$
	\item \begin{rawDef}
		\label{torus}
		IDer n-dimensionale Torus $T^n=S^1\times\cdots\times S^1$ ist eine n-dimensionale Mannigfaltigkeit die \hyperref[homoemorph]{homöomorph} zur $S^n$ ist.
	\end{rawDef}
\end{enumerate}
\end{rawDef}